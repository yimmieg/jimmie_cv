%% start of file `jdoe_classic.tex'.
%% Copyright 2006 Xavier Danaux.
%
% This work may be distributed and/or modified under the
% conditions of the LaTeX Project Public License version 1.3c,
% available at http://www.latex-project.org/lppl/.


\documentclass[10pt]{moderncv}

% moderncv styles
%\moderncvstyle{casual}       % optional argument are 'nocolor' (black & white cv) and 'roman' (for roman fonts, instead of sans serif fonts)
\moderncvstyle{classic}       % idem

% character encoding
%\usepackage[utf8]{inputenc}   % replace by the encoding you are using

% personal data (the given example is exhaustive; just give what you want)
\firstname{Dimitrios}
\familyname{I.~Diochnos}
\title{Computer Scientist}
%\address{17 Spartis Street\\144-52 Metamorphosis\\Athens, Attika, Hellas}  % for classic style
\address{Office 1211\\ Science and Engineering Offices\\ 851 South Morgan street\\ Chicago, IL 60606, U.S.A}  % for classic style
%\address{12 somestreet, 3456 somecity} % for casual style
%\phone{+30 210 28 49 173\\+30 694 20 59 566}
\phone{+1 312 413 8263}
\email{diochnos [AT] math.uic.edu}
\extrainfo{\weblink{http://www.di.uoa.gr/\textasciitilde stud1098}}
%\photo[84pt]{dim2} % also optional, and the optional argument is the height the picture must be resized to
\quote{Any intelligent fool can make things bigger, more complex, and more violent. It takes a touch of genius -- and a lot of courage -- to move in the opposite direction.}% also optional

%\renewcommand{\listsymbol}{{\fontencoding{U}\fontfamily{ding}\selectfont\tiny\symbol{'102}}} % define another symbol to be used in front of the list items

% the ConTeXt symbol
\def\ConTeXt{%
  C%
  \kern-.0333emo%
  \kern-.0333emn%
  \kern-.0667em\TeX%
  \kern-.0333emt}

% slanted small caps (only with roman family; the sans serif font doesn't exists :-()
%\usepackage{slantsc}
%\DeclareFontFamily{T1}{myfont}{}
%\DeclareFontShape{T1}{myfont}{m}{scsl}{ <-> cork-lmssqbo8}{}
%\usefont{T1}{myfont}{m}{scsl}Testing the font

% command and color used in this document, independently from moderncv 
\definecolor{see}{rgb}{0.5,0.5,0.5}% for web links
\newcommand{\up}[1]{\ensuremath{^\textrm{\scriptsize#1}}}% for text subscripts

%----------------------------------------------------------------------------------
%            content
%----------------------------------------------------------------------------------
\begin{document}
\maketitle
%\makequote
\section{\textsc{Personal Information}}
\cvitem{Father's Name}{Ioannis}
\cvitem{Date of Birth}{April 13, 1980}
\cvitem{Place of Birth}{Cholargos, Attika, Hellas}
\cvitem{Nationality}{Hellenic}
\cvitem{Marital Status}{Single}
%\cvitem{Boston Univ. ID}{\texttt{U82115670}}

\section{\textsc{Education}}
\cventry{2007 -- now}{Working for my Ph.D.~in Mathematical Computer Science.}{Department of Mathematics, Statistics, and Computer Science}{University of Illinois at Chicago}{USA. \newline %\hspace{\fill}
{Home: {\color{see} \weblink{http://www.math.uic.edu}}}}{\hspace{\fill}}
%}{}
%\cvitem{}{\textsc{Master Thesis}}
%\cvitem{title}{\emph{Real Solving on Algebraic Systems of Small Dimension}}
%\cvitem{supervisors}{Professors Ioannis Z. Emiris, Elias Koutsoupias and Evagelos Raptis}
%\cvitem{description}{\small Algorithms for real solving of polynomial
%  systems of small dimension via Sturm sequences. An algebraic library in
%  Maple has been created as part of the implementation.}
%\cvitem{}{}
\cventry{2007}{M.Sc. in Logic, Theory of Algorithms and Computation}{Department of Mathematics}{National and Kapodistrian University of Athens}{Hellas. \newline %\hspace{\fill}
{Home: {\color{see} \weblink{http://mpla.math.uoa.gr}}}}{GPA: $8.4$ out of $10.0$\hspace{\fill}}
\cvitem{}{\textsc{Master Thesis}}
\cvitem{title}{\emph{Real Solving on Algebraic Systems of Small Dimension}}
\cvitem{supervisors}{Professors Ioannis Z. Emiris, Elias Koutsoupias and Evagelos Raptis}
\cvitem{description}{\small Algorithms for real solving of polynomial
  systems of small dimension via Sturm sequences. An algebraic library in
  Maple has been created as part of the implementation.}
\cvitem{}{}
\cventry{2004}{Ptychion (4-year Bachelor) in Computer Science}{Department of Informatics and Telecommunications}{National and Kapodistrian University of Athens}{Hellas. \newline %\hspace{\fill} 
{Home: {\color{see} \weblink{http://www.di.uoa.gr}}}}{GPA: $7.4$ out of $10.0$}
\cvitem{}{\textsc{Undergraduate Thesis}}
\cvitem{title}{\emph{Application of Reinforcement Learning and Combinatorial Search to One-Player Games}}
\cvitem{supervisors}{Professor Panagiotis Stamatopoulos}
\cvitem{description}{\small Augmenting learning process of classical \textit{reinforcement-learning} agents through combinatorial search techniques and an application in game \textit{Solo}.}

\section{\textsc{Scholarships}}
\cvitem{Undergraduate}{I fulfilled my undergraduate studies under
  scholarship by "Zossima Brothers" %($A\varphi\acute{\omega}\nu \ Z\omega\sigma\iota\mu\acute{\alpha}$)
foundation.}
%\cvitem{Empodia}{Most interesting problem in IOI-2004. See articles section for more information.}

\vspace*{-1cm}
\section{\textsc{Languages}}
\cvitem{Greek}{Fluent\hspace{2em}(mother tongue)}
\cvitem{English}{Cambridge First Certificate in English, Dec 1994}
\cvitem{German}{Goethe-Institut Zertifikat Deutsch als Fremdsprache, May 1995.}

%\section{\textsc{Awards - Scholarships}}
%\cvitem{Undergraduate}{I fullfilled my undergraduate studies under scholarship by "Zossima Brothers" ($A\varphi\acute{\omega}\nu \ 
%Z\omega\sigma\iota\mu\acute{\alpha}$) foundation.}

\closesection{}
\pagebreak{}


\section{\textsc{Standardized Tests}}
\cvitem{}{\textbf{iBT TOEFL Scores}}
\cvitem{Test Date}{Nov 17, 2006}
\cvitem{Reading}{26/30}
\cvitem{Listening}{27/30}
\cvitem{Speaking}{23/30}
\cvitem{Writing}{30/30}
\cvitem{Total}{106/120}
\cvitem{Equivalent Scores}{CBT: 263/300, PBT: 623/677}
\cvitem{}{\textbf{CBT GRE General Scores}}
\cvitem{Test Date}{Dec 14, 2006}
\cvitem{Quantitative}{800/800 \hspace{1.5cm} Percentile Rank: 94\%}
\cvitem{Verbal}{360/800 \hspace{1.5cm} Percentile Rank: 20\%}
\cvitem{Analytical Writing}{4.5/6.0 \hspace{1.65cm} Percentile Rank: 52\%}

%\section{\textsc{Standardized Tests}}\closesection
%\begin{center}
%\begin{tabular}{|c|c|c|c|c|c|c|}\hline
%TOEFL & Exam Date & Reading & Listening & Speaking & Writing & Total \\ \hline
%iBT & 11/17/2006 & 26 & 27 & 23 & 30 & 106 \\ \hline
%\end{tabular}
%\end{center}




\section{\textsc{Scientific Activities}}
\cvitem{2006 -- 2007\\ Jan \ \ \ \ \ Jul}{Member of the \textit{Lab of Geometric and Algebraic Algorithms} at the Dept. of Informatics and Telecommunications, University of Athens.\\Homepage: {\color{see} \weblink{http://www.di.uoa.gr/\textasciitilde erga}}}
%\weblink{\texttt{http://www.di.uoa.gr/\textasciitilde erga}}}
\cvitem{Sep 2004}{Member of the International Scientific Committee (\texttt{ISC}) at the International Olympiad in Informatics (\texttt{IOI-2004}) that was held in Athens.\\Homepage: {\color{see} \weblink{http://www.ioi2004.org}}}


\section{\textsc{Scientific Interests}}
%\cvitem{Theoretical Informatics}{Computer Algebra, Cryptography, Algorithmic Number Theory, Computational Geometry, Numerical Analysis, Interval Arithmetic}
%\cvitem{Computer Systems}{Dynamic Programming, Reinforcement Learning}
\cvlistitem{Cryptography - Information Security}
%\cvlistitem{Computer Security}
\cvlistitem{Computer Algebra}
%\cvlistitem{Computational Algebraic Geometry}
\cvlistitem{Algorithmic Number Theory}
\cvlistitem{Computational Geometry}
\cvlistitem{Randomized Algorithms}
\cvlistitem{Numerical Analysis}
%\cvlistitem{Interval Arithmetic}
\cvlistitem{Reinforcement Learning}
%\cvlistitem{Dynamic Programming}


\section{\textsc{Employment History}}
\cvitem{Aug 2007 -- now}{I wok as a Teaching Assistant at University of Illinois at Chicago. My discussion sessions for this semester are related to Math 118: Mathematical Reasoning.}
\cvitem{Jan -- Jun 2002}{I worked as a teacher in lectures given to high-school teachers under the program \textit{Lifelong Learning: Familiarization with new technologies} which was supported by the Hellenic Ministry of Education due to the general development project "Information Society".}
\cvitem{Jan -- Jul 2000}{I worked at \textit{Othisi}
%($'\Omega\vartheta\eta\sigma\eta$ \texttrademark) 
as a Computer Science teacher for the course \textit{Developing Applications under a Programming Environment}.}
\cvitem{1998 -- 1999\\Oct \ \ \ \ \ Jul}{Head of the Mathematics department at the \textit{Students' Learning Support Center} at \textit{Othisi}.
%($'\Omega\vartheta\eta\sigma\eta$ \texttrademark)
%which prepares high-school students ... }
}
\cvitem{Othisi}{\textit{Othisi} is a preparatory institute which prepares high-school students for entering Higher-Level Education \{Universities, Technological Institutions\}.\\
Contact Information: \\
Mitropoleos \& Chaimanta 7, 151-25 Marousi, Attika, Hellas\\ Phone: +30-210-61.28.814, +30-210-61.43.812\\ Fax: +30-210-80.61.353\\ Webpage: {\color{see} \weblink{http://www.othisi.gr}}}

\closesection{}
\pagebreak{}

\section{\textsc{Working Experience}}
%\cvitem{\textbf{Operating Systems}}{}
\cvitem{}{\textbf{Operating Systems}}
\cvitem{}{All Microsoft\texttrademark\ operating systems, Solaris\texttrademark\ Unix, MacOS\texttrademark, Linux}
%\cvitem{Procedural Programming}{C, Perl, Pascal, QBasic, Fortran}
%\cvitem{}{}
%\cvitem{\textbf{Programming Languages}}{}
\cvitem{}{\textbf{Programming Languages}}
\cvitem{Procedural}{C, Perl, Pascal, QBasic, Fortran}
%\cvitem{Object-Oriented Programming}{C++, Java, Object Pascal}
\cvitem{Object-Oriented}{C++, Java, Object Pascal}
%\cvitem{Visual Programming}{Visual C++, Visual Basic}
\cvitem{Visual}{Visual C++, Visual Basic}
%\cvitem{Interpreted Programming}{GW-Basic, Logo}
\cvitem{Interpreted}{GW-Basic, Logo}
%\cvitem{Logic Programming}{LPA-Prolog}
\cvitem{Logic}{LPA-Prolog}
%\cvitem{Functional Programming}{Haskell}
\cvitem{Functional}{Haskell}
%\cvitem{Shells - Script Programming}{C, Bourne, Korn, bash, z, pk, MS-DOS, VBScript, JavaScript, HTML}
\cvitem{Shells-Scripts}{C, Bourne, Korn, bash, z, pk, MS-DOS, VBScript, JavaScript}
\cvitem{Web}{HTML}
%\cvitem{}{}
%\cvitem{\textbf{Tools}}{}
\cvitem{}{\textbf{Tools}}
\cvitem{Administration}{Apache}
%\cvitem{Data-Base Tools}{Oracle SQLPlus\texttrademark, Microsoft SQLServer\texttrademark}
\cvitem{Data-Base}{Oracle SQLPlus\texttrademark, Microsoft SQLServer\texttrademark}
%\cvitem{Mathematical Tools}{Maple\texttrademark, MatLab\texttrademark, GNUPlot, Graphmat}
\cvitem{Scientific}{Maple\texttrademark, MatLab\texttrademark, GNUPlot, Graphmat}
%\cvitem{Office Automation}{\LaTeX, Microsoft Office\texttrademark}
%\cvitem{}{}
%\cvitem{\textbf{Office Automation}}{}
\cvitem{}{\textbf{Office Automation}}
\cvitem{}{\TeX, \LaTeX, Microsoft Office\texttrademark}

%\section{\textsc{Working Experience}}
%\cvitem{Operating Systems}{All Microsoft\texttrademark\ operating systems, Solaris\texttrademark\ Unix, MacOS, Linux}
%\cvitem{Programming}{}
%\cvlistdoubleitem{Procedural}{C, Perl, Pascal, QBasic, Fortran}
%\cvlistdoubleitem{Object-Oriented}{C++, Java, Object Pascal}
%\cvlistdoubleitem{Visual}{Visual C++, Visual Basic}
%\cvlistdoubleitem{Interpreted}{GW-Basic, Logo}
%\cvlistdoubleitem{Logic}{LPA-Prolog}
%\cvlistdoubleitem{Functional}{Haskell}
%\cvlistdoubleitem{Shells-Scripts}{C, Bourne, Korn, bash, z, pk, MS-DOS, VBScript, JavaScript}
%\cvlistdoubleitem{Web}{HTML}
%\cvitem{Tools}{}
%\cvlistdoubleitem{Scientific}{Maple\texttrademark, MatLab\texttrademark, GNUPlot, Graphmat}
%\cvitem{Data-Base Tools}{Oracle SQLPlus\texttrademark, Microsoft SQLServer\texttrademark}
%\cvlistdoubleitem{Data-Base}{Oracle SQLPlus\texttrademark, Microsoft SQLServer\texttrademark}
%\cvitem{Office Automation}{\LaTeX, Microsoft Office\texttrademark}
%\cvitem{Office Automation}{\LaTeX, Microsoft Office\texttrademark}

%\section{Experience}
%\cventry{February 2006--\\current}{Maintainer of the a CTAN package}{CTAN}{World}{}{Maintainer of the {\ttfamily moderncv} package, meant to ease the production of beautiful curriculum vit\ae{}s.}
%\cventry{2005--2006}{Mathematics tutor}{UCL}{Louvain-la-Neuve}{}{Supervision of practical sessions for a mathematical course given to second year engineering students (course \emph{FSAB1104: Numerical Methods}).\hfill{\itshape\color{see}\footnotesize{}See \httplink{www.legat-online.be/b2q1/num}.}}
%\cventry{2004--2006}{Cultural project leader}{Tchouque-Tschouk Kot}{Louvain-la-Neuve}{}{Leader of a student home with a cultural project, requiring day to day management as well as the organization of public events.\hfill{\itshape\color{see}\footnotesize{}See \httplink{www.organe.be}.}}
%\cventry{1999--2001}{IMO preselected}{SBPMef}{Wépion}{}{Advanced mathematical training, as Belgian preselected candidate for the International Mathematical Olympiads, selected by the Belgian mathematical society.\hfill{\itshape\color{see}\footnotesize{}See \weblink{imo.math.ca/belgium.html}.}}


%\closesection{}
%\pagebreak{}







%\section{Section with a list}
%\cvlistitem{Single item.}
%\cvlistitem{Another single item.}
%\cvlistdoubleitem{Double\dots{}}{\dots{} item.}
%\cvlistdoubleitem{Another double\dots{}}{\dots{} item.}

%\section{Section with your own content}\closesection
%Your content here, inside the normal \LaTeX{} environment. You can use any regular \LaTeX{} command, display mathematics
%\[e =m\,c^2,\]
%put some table or figure, \dots

%\emptysection{}
%\cvitem{Now}{Back to moderncv layout, without making a new section :-)}



\section{\textsc{Software}}
\cvitem{}{The following applications are freely available through my homepage:}
%
% Academic
%
\cvitem{Academic}{\textbf{Database for Undergraduate Lessons}\\ This
  is a program that can be used as a database for undergraduate
  lessons passed at the Dept. of Informatics and Telecommunications as
  well as a tool for statistical analysis of the GPA and other
  departmental parameters which are crucial for graduate applications.}
%
% Computational Biology
%
%\cvitem{}{}
\cvitem{Computational Biology}{\textbf{Inversion Distance and Sorting
    by Reversals}\\ Tools that compute the inversion distance of two
    genomes as well as perform sorting by reversals between two
    genomes. Part of the source code was used as testbed in \texttt{IOI-2004}.}
%
% Computational Algebraic Geometry
%
%\cvitem{}{}
\cvitem{Computational Alg. Geometry}{\textbf{Gr\"obner Bases Computation}\\\textit{Symbolic Algebra} is an application in C++ that computes a Gr\"obner base of a system of polynomials with integer coefficients. Source code requires MS Visual Studio $6.0$ and is currently running only under Windows OS. The homepage of this application is in greek and can be found here: {\color{see} \weblink{http://www.di.uoa.gr/\textasciitilde stud1098/Symbolic\_Algebra}}\\Current version is $0.9$.}
%
% Computational Geometry
%
%\cvitem{}{}
\cvitem{Computational Geometry}{\textbf{Voronoi Diagrams and Delaunay
    Triangulations}\\A program (in Visual Basic) that computes the Voronoi Diagram and
    the Delaunay Triangulation of points in the euclidean plane.}
%
% Computer Algebra
%
%\cvitem{}{}
\cvitem{Computer Algebra}{\textbf{SLV Maple Library}\\
    %This is the project I am currently working on (jointly
    %with Professor I.~Z.~Emiris and Dr.~E.~P.~Tsigaridas) in my master
    %thesis. The library is developed in Maple and also provides an
    %implementation for Real Algebraic Numbers. The homepage is accessible at:
    SLV is a library used in Maple\texttrademark. The acronym comes from Sturm soLVer. It was developed as part of my master's thesis
    and solves univariate polynomials or bivariate polynomial systems using Sturm sequences. The solutions are (pairs of) Real Algebraic Numbers
    in Isolating Interval Representation. The homepage of the library is accessible at:
   {\color{see} \weblink{http://www.di.uoa.gr/\textasciitilde erga/soft/SLV\_index.html}}}
%
% Cryptography
%
%\cvitem{}{}
\cvitem{Cryptography}{\textbf{Discrete Logarithm}\\Some tools (source code and win32 executables) that attack the discrete logarithm problem when machine precision suffices. The webpage hosting these tools is available only in greek at {\color{see} \weblink{http://www.di.uoa.gr/\textasciitilde stud1098/cryptography/discrete\_log.html}}\\
Source code is easily extensible in the generic case with the use of GNU Multiprecision Arithmetic Library (GMP).}
%\cventry{Cryptography}{Attacking Discrete Logarithm}{Some tools (source code and win32 executables) that attack the discrete logarithm problem in easy cases}{webpage: {\color{see} \weblink{http://www.di.uoa.gr/\textasciitilde stud1098/cryptography/discrete\_log.html}}}{May 2005}{Language: Greek}
\cvitem{}{\textbf{Elliptic Curves}\\Tools for operations on
  elliptic curves. Online is available integer factorization with the use of
  elliptic curves.}
%
% Optimization.
%
\cvitem{Optimization}{\textbf{The Ellipsoid Method}\\The popular Ellipsoid Method used in Linear Programming, implemented in C.}
%
%
%
\cvitem{Reinforcement Learning}{\textbf{Optimal Policy in game Solo}\\
An RL-agent that finds optimal policy in game Solo. The learining process is augmented through combinatorial search techniques. [Undergraduate Thesis]}
%
% Strategy Games
%
%\cvitem{}{}
\cvitem{Strategy Games}{\textbf{Heroes of Might and Magic III}}
\cvitem{}{The official homepage for the generic problem of Skill Advancing, which also hosts solvers for the problems presented below, can be found here: {\color{see} \weblink{http://heroescommunity.com/viewthread.php3?TID=17812\&pagenumber=1}}}
\cvitem{}{\textit{\texttt{internals\_mc:} Evaluating Policies with Monte Carlo methods in Skill-Selection problem}, dimis, July 2007.
Current version is $1.1$ and supports five popular deterministic policies. Soon more will be available.}
\cvitem{}{\textit{\texttt{ansa, ansa\_dnf:} Solver for ANSA problem}, dimis, April 2006.
Source code for \texttt{ansa} is also available in GNU Multiprecision Arithmetic Library (GMP), although it is not necessary for practical applications. \texttt{ansa\_dnf} was developed in July 2006 in order to reply to more interesting questions posed in DNF form.}

%\closesection{}
%\pagebreak{}

\section{\textsc{Publications}}
\cvitem{}{2007}
\cvitem{Computer Algebra}{\textit{On the complexity of real solving bivariate systems}, Dimitrios I.~Diochnos, Ioannis Z.~Emiris, and Elias P.~Tsigaridas, Proc.~ Annual ACM International Symposium on Symbolic and Algebraic Computation (ISSAC), Waterloo, Canada, 2007. A copy is accessible at\\ 
{\color{see} \weblink{http://www.di.uoa.gr/\textasciitilde stud1098/Computer\_Algebra/det-issac-07.pdf}}}

\section{\textsc{Theses, Technical Reports and other Articles}}
%
% Computer Algebra.
%
%\cvitem{}{\textsc{Submitted}}
%\cvitem{Computer Algebra}{\textit{On the complexity of real solving bivariate systems}, Dimitrios I.~Diochnos, Ioannis Z.~Emiris, and Elias P.~Tsigaridas, ISSAC 2007. A copy is accessible at\\ 
%{\color{see} \weblink{http://www.di.uoa.gr/\textasciitilde stud1098/Computer\_Algebra/det-issac-07.pdf}}}
%\cvitem{Computational Geometry}{\textit{Operations on real plane algebraic curves}, Dimitrios I. Diochnos, Ioannis Z. Emiris and Elias P. Tsigaridas. To be submitted EWCG 2007. \\ A draft is available online through my webpage: \color{see} \weblink{http://www.di.uoa.gr/\textasciitilde stud1098}}
%
% Master Thesis
% Computer Algebra
%
\cvitem{}{\textsc{Master Thesis}}
\cvitem{Computer Algebra}{\emph{Real Solving on Algebraic Systems of Small Dimension}, Dimitrios I.~Diochnos, Master Thesis, June 2007. Copy available at\\
{\color{see} \weblink{http://www.di.uoa.gr/\textasciitilde stud1098/diploma/diploma\_diochnos.pdf}}}
%
% Undergraduate Thesis
% Reinforcement Learning
%
\cvitem{}{\textsc{Undergraduate Thesis}}
\cvitem{Reinforcement Learning}{\emph{Application of Reinforcement Learning and Combinatorial Search to One-Player Games}, Dimitrios I.~Diochnos, Undergraduate Thesis, Feb 2004. Copy available in greek at\\
{\color{see} \weblink{http://www.di.uoa.gr/\textasciitilde stud1098/SOLO/diochnos\_undergrad.pdf}}}
%
% Technical Reports.
%
\cvitem{}{\textsc{Technical Reports}}
%
% INRIA
%
\cvitem{Computer Algebra}{\textit{On the complexity of real solving bivariate systems}, D.~I.~Diochnos, I.~Z.~Emiris, and E.~P.~Tsigaridas. INRIA RR 6116. Available at\\
{\color{see} \weblink{https://hal.inria.fr/inria-00129309}}}
%
% Surveys and Other Reports.
%
\cvitem{}{\textsc{Surveys and Other Articles}}
%
% IOI-2004
%
\cvitem{IOI-2004}{\textit{Enumerating Hurdles}, Dimitrios I.~Diochnos and Ioannis Z.~Emiris, problem in \texttt{IOI-2004}, Sep 2004. An electronic copy of the proposition is available at\\ {\color{see} \weblink{http://www.di.uoa.gr/\textasciitilde stud1098/ioi2004/hurdles.pdf}}.}
%
% Cryptography.
%
%\cvitem{}{}
\cvitem{Cryptography}{\textit{Ways to Attack Popular Cryptosystems}, Dimitrios I.~Diochnos, July 2005. Copy available in greek at\\
{\color{see} \weblink{http://www.di.uoa.gr/\textasciitilde stud1098/cryptography/Attacks/attacks.pdf}}}
%
% Computational Algebraic Geometry
%
\cvitem{Computational Alg. Geometry}{\textit{An Upper Bound on the Complexity of the Buchberger Algorithm}, Dimitrios I.~Diochnos and George Piliouras, $\mu\prod\lambda\forall$, Feb 2005. Copy available in greek at\\
{\color{see} \weblink{http://www.di.uoa.gr/\textasciitilde stud1098/Symbolic\_Algebra/buch\_complexity.zip}}}
%
% Additional space?
%
\cvitem{}{}
\cvitem{}{}
%
% Other Articles.
%
\cvitem{}{\textsc{Other Articles}}
%
% Programming
%
\cvitem{Programming}{\textit{Introduction to Programming ( with C )}, is a thread which aims to cover basic algorithmic techniques with a presentation of the C programming language. The nature of the thread is much more interactive than other tutorials online, since it can be found on a public forum. The homepage is: {\color{see} \weblink{http://heroescommunity.com/viewthread.php3?TID=20082}}}
%
% Strategy Games.
%
\cvitem{Strategy Games}{The following articles were published in HeroesCommunity, hence the tone is more informal compared to a technical report. Nevertheless, they have influenced contemporary practice:}
\cvitem{}{\textit{Crag-Hack on ANSA}, dimis, July 2006.\\ 
{\color{see} \weblink{http://www.di.uoa.gr/\textasciitilde stud1098/HeroesIII/internals/crag-hack/crag-ansa.pdf}}}
\cvitem{}{\textit{Imp Caches, Statistical Analysis}, dimis, June 2005. Electronic copy:\\
{\color{see} \weblink{http://www.di.uoa.gr/\textasciitilde stud1098/HeroesIII/Caches/Experiment/caches\_statistics.pdf}}}



%
% Other Activities.
%
\section{\textsc{Other Activities}}
\cvitem{Feb 2005--Today}{Member of HeroesCommunity. My nickname is \texttt{dimis} and my profile can be found here: {\color{see} \weblink{http://heroescommunity.com/member.php3?action=viewprofile\&UserName=dimis}}}
\cvitem{Jun 1999--Today}{Member of SETI@HOME. Homepage: {\color{see} \weblink{http://setiathome.berkeley.edu}} My profile can be found here: {\color{see} \weblink{http://setiathome.berkeley.edu/view\_profile.php?userid=2001284}}}
%
% Other Interests.
%
%\section{\textsc{Other Interests}}
%\cvitem{}{}
%\cvitem{}{}
\cvitem{}{\textsc{Other Interests}}
\cvitem{Education}{I am interested in making science and technology available to the public, especially to people with disabilities.}
\cvitem{Strategy Games}{I like all sort of turn-based strategy games. More particularly Heroes of Might and Magic III, Sid Meier's Civilization II and ofcourse chess.}
\cvitem{Hill Climbing}{I always find intriguing to climb up a mountain.}



%\closesection{}
%\pagebreak{}


\section{\textsc{References}}%\closesection
\cvitem{}{These persons are familiar with my professional qualifications and my character:}
%\cvitem{}{} % Empty line.
%
% Emiris
%
%\cvitem{}{\begin{tabular}{@{}lll@{}}
%\textbf{Associate Professor Dr. Ioannis Z. Emiris \ \ \ \ \ \ \ \ \ \ \ \ } \\
%\textbf{Professor Dr. Ioannis Z. Emiris \ \ \ \ \ \ \ \ \ \ \ \ \ \ \ \ \ \ \ \ \ \ \ \ \ \ } \\
%Master thesis supervisor & Phone: & +30-210-727.5105\\
%Dept. of Informatics and Telecommunications & Fax: & +30-210-727.5114\\
%University of Athens & Email: & emiris@di.uoa.gr \\
%Panepistimiopolis, 15784 Athens, Hellas \\
%\end{tabular}}
%
% Koutsoupias
%
%\cvitem{}{\begin{tabular}{@{}lll@{}}
%\textbf{Professor Dr. Elias Koutsoupias \ \ \ \ \ \ \ \ \ \ \ \ \ \ \ \ \ \ \ \ \ \ \ \ } \\
%Dept. of Informatics and Telecommunications & Phone: & +30-210-727.5122\\
%University of Athens & Fax: & +30-210-727.5114\\
%Panepistimiopolis, 15784 Athens, Hellas & Email: & elias@di.uoa.gr \\
%\end{tabular}}
%
% Raptis
%
\cvitem{}{\begin{tabular}{@{}lll@{}}
\textbf{Professor Ioannis Z.~Emiris} & \\
Master thesis supervisor & Phone: & +30-210-727.5105\\
Dept.~of Informatics and Telecommunications \ & Fax: & +30-210-727.5114\\
University of Athens & Email: & emiris [AT] di.uoa.gr \\
Panepistimiopolis, 15784 Athens, Hellas \\
 & \\
\textbf{Professor Elias Koutsoupias} & \\
Dept.~of Informatics and Telecommunications \ & Phone: & +30-210-727.5122\\
University of Athens & Fax: & +30-210-727.5114\\
Panepistimiopolis, 15784 Athens, Hellas & Email: & elias [AT] di.uoa.gr \\
 & \\
\textbf{Professor Evagelos Raptis} & \\
Dept.~of Mathematics & Phone: & +30-210-727.6347\\
University of Athens \ & Fax: & +30-210-727.6378\\
Panepistimiopolis, 15784 Athens, Hellas & Email: & eraptis [AT] math.uoa.gr \\ 
 & \\
\textbf{Professor Panagiotis Stamatopoulos \ } & \\
Undergraduate thesis supervisor & Phone: & +30-210-727.5202\\
Dept.~of Informatics and Telecommunications \ & Fax: & +30-210-727.5214\\
University of Athens & Email: & takis [AT] di.uoa.gr \\
Panepistimiopolis, 15784 Athens, Hellas \\
\end{tabular}}
%
% Stamatopoulos
%
%\cvitem{}{\begin{tabular}{@{}lll@{}}
%\textbf{Assistant Professor Dr. Panagiotis Stamatopoulos \ \ } \\
%Undergraduate thesis supervisor & Phone: & +30-210-727.5202\\
%Dept. of Informatics and Telecommunications & Fax: & +30-210-727.5214\\
%University of Athens & Email: & takis@di.uoa.gr \\
%Panepistimiopolis, 15784 Athens, Hellas \\
%\end{tabular}}



% Use the following when references are alone on last page.
% The following is for use when Elias Koutsoupias can be referenced.:
%\vspace{14cm}

%%\vspace{15.7cm}



%%\vspace{13.5cm}
%%\vspace{13cm}
%%\vspace{12cm}

%\vspace{8.5cm}
\vspace*{\fill}

\section{\textsc{Updated}}
%\cvitem{}{March 29, 2007}
\cvitem{}{September 22, 2007}

%\nocite{*}
%\bibliographystyle{plain}
%\bibliography{jdoe_publications}


\end{document}


%% end of file `diochnos_cv.tex'.
